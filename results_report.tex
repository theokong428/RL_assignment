% =============================================================================
% Computational Results — Parts 2 & 3
% Paste this directly into your main LaTeX document
% Requires: amsmath, amssymb, booktabs packages
% =============================================================================

\subsection{Aggregation Summary}

The original 440 customer postcode districts are aggregated into postal area zones (e.g.\ AB, DD, EH), reducing the customer dimension to approximately 15 aggregated super-customers. Each zone's demand is the sum of its constituent districts, and the representative location is the centroid of the district coordinates. Additionally, the candidate warehouse set is filtered to the top-5 by capacity within each postal area, reducing the 440 candidates to 75.


% =============================================================================
\subsection{Part 2: Multi-Period Deterministic MECWLP}
% =============================================================================

% -----------------------------------------------------------------------------
\subsubsection{Question (1): MILP Formulation}
% -----------------------------------------------------------------------------

We extend the single-period MECWLP to a multi-period setting with $T$ time periods. Warehouses, once built, remain open for the remainder of the planning horizon.

\paragraph{Sets.}
\begin{itemize}
    \item $\mathcal{I}$: set of (aggregated) customers (postal area zones)
    \item $\mathcal{J}$: set of candidate warehouse locations
    \item $\mathcal{K}$: set of suppliers
    \item $\mathcal{P}$: set of product groups
    \item $\mathcal{T} = \{1, \dots, T\}$: set of time periods
    \item $\mathcal{K}_p \subseteq \mathcal{K}$: suppliers of product group $p$
\end{itemize}

\paragraph{Parameters.}
\begin{itemize}
    \item $f_j$: one-time setup cost for warehouse $j$
    \item $g_j$: annual operating cost for warehouse $j$ (incurred each period it is open)
    \item $u_j$: capacity of warehouse $j$ (kg)
    \item $s_k$: capacity of supplier $k$ (kg per period)
    \item $b_{i,p,t}$: demand of customer zone $i$ for product $p$ in period $t$ (kg)
    \item $c^{\text{down}}_{j,i}$: unit transport cost from warehouse $j$ to customer $i$ (per kg, round-trip, 3.5t van)
    \item $c^{\text{up}}_{k,j}$: unit transport cost from supplier $k$ to warehouse $j$ (per kg, round-trip)
\end{itemize}

\paragraph{Decision Variables.}
\begin{itemize}
    \item $y_{j,t} \in \{0,1\}$: 1 if warehouse $j$ is \emph{built} (constructed) in period $t$
    \item $z_{j,t} \in \{0,1\}$: 1 if warehouse $j$ is \emph{open} (operational) in period $t$
    \item $x_{j,i,p,t} \in [0,1]$: fraction of customer $i$'s demand for product $p$ in period $t$ served by warehouse $j$
    \item $w_{k,j,t} \geq 0$: quantity (kg) shipped from supplier $k$ to warehouse $j$ in period $t$
\end{itemize}

\paragraph{Formulation.}
\begin{align}
\min \quad & \sum_{j \in \mathcal{J}} \sum_{t \in \mathcal{T}} f_j \, y_{j,t}
          + \sum_{j \in \mathcal{J}} \sum_{t \in \mathcal{T}} g_j \, z_{j,t} \notag \\
          & + \sum_{j \in \mathcal{J}} \sum_{i \in \mathcal{I}} \sum_{p \in \mathcal{P}} \sum_{t \in \mathcal{T}} c^{\text{down}}_{j,i} \, b_{i,p,t} \, x_{j,i,p,t}
          + \sum_{k \in \mathcal{K}} \sum_{j \in \mathcal{J}} \sum_{t \in \mathcal{T}} c^{\text{up}}_{k,j} \, w_{k,j,t}
\label{eq:part2-obj}
\end{align}
subject to:
\begin{align}
& z_{j,t} = \sum_{\tau=1}^{t} y_{j,\tau}
  & \forall\, j \in \mathcal{J},\; t \in \mathcal{T}
  \label{eq:c1} \tag{C1} \\[4pt]
& \sum_{t \in \mathcal{T}} y_{j,t} \leq 1
  & \forall\, j \in \mathcal{J}
  \label{eq:c2} \tag{C2} \\[4pt]
& \sum_{j \in \mathcal{J}} x_{j,i,p,t} = 1
  & \forall\, i \in \mathcal{I},\; p \in \mathcal{P},\; t \in \mathcal{T}
  \label{eq:c3} \tag{C3} \\[4pt]
& \sum_{i \in \mathcal{I}} \sum_{p \in \mathcal{P}} b_{i,p,t} \, x_{j,i,p,t} \leq u_j \, z_{j,t}
  & \forall\, j \in \mathcal{J},\; t \in \mathcal{T}
  \label{eq:c5} \tag{C4} \\[4pt]
& \sum_{k \in \mathcal{K}_p} w_{k,j,t} = \sum_{i \in \mathcal{I}} b_{i,p,t} \, x_{j,i,p,t}
  & \forall\, j \in \mathcal{J},\; p \in \mathcal{P},\; t \in \mathcal{T}
  \label{eq:c6} \tag{C5} \\[4pt]
& \sum_{j \in \mathcal{J}} w_{k,j,t} \leq s_k
  & \forall\, k \in \mathcal{K},\; t \in \mathcal{T}
  \label{eq:c7} \tag{C6} \\[4pt]
& y_{j,t} \in \{0,1\},\; z_{j,t} \in \{0,1\}
  & \forall\, j \in \mathcal{J},\; t \in \mathcal{T}
  \notag \\
& x_{j,i,p,t} \in [0,1],\; w_{k,j,t} \geq 0
  & \forall\, j, i, p, t, k
  \notag
\end{align}

\noindent\textbf{Explanation of constraints:}
\begin{itemize}
    \item \eqref{eq:c1} \textbf{Cumulative opening}: $z_{j,t}$ equals the cumulative sum of build decisions up to period $t$, ensuring that once warehouse $j$ is built, it remains open for all subsequent periods.
    \item \eqref{eq:c2} \textbf{Build at most once}: each warehouse can be constructed in at most one period.
    \item \eqref{eq:c3} \textbf{Demand satisfaction}: each customer's full demand for every product must be served in every period.
    \item \eqref{eq:c5} \textbf{Warehouse capacity}: the total demand served by warehouse $j$ in period $t$ cannot exceed its capacity, and only open warehouses ($z_{j,t}=1$) can serve demand.
    \item \eqref{eq:c6} \textbf{Flow balance}: the total inbound supply of product $p$ to warehouse $j$ must equal the total outbound demand for product $p$ fulfilled by warehouse $j$.
    \item \eqref{eq:c7} \textbf{Supplier capacity}: each supplier has a per-period capacity limit.
\end{itemize}


% -----------------------------------------------------------------------------
\subsubsection{Question (2): Implementation and Results}
% -----------------------------------------------------------------------------

\paragraph{Model Size.}

\begin{table}[h]
\centering
\caption{Model dimensions for Part~2}
\label{tab:part2-dimensions}
\begin{tabular}{lr}
\toprule
Component & Count \\
\midrule
Binary variables $y_{j,t}$ (build decisions) & 4{,}400 \\
Binary variables $z_{j,t}$ (open indicators) & 4{,}400 \\
Continuous variables $x_{j,i,p,t}$ (allocation fractions) & 264{,}000 \\
Continuous variables $w_{k,j,t}$ (supply flows) & 233{,}200 \\
\textbf{Total variables} & \textbf{506{,}000} \\
Total constraints & 27{,}970 \\
\bottomrule
\end{tabular}
\end{table}

\paragraph{Solution Quality.}
The model was solved using FICO Xpress~v9.7.0 with a time limit of 600 seconds.

\begin{table}[h]
\centering
\caption{Part~2 solution summary}
\label{tab:part2-solution}
\begin{tabular}{lr}
\toprule
Metric & Value \\
\midrule
Status & Feasible \\
Objective value & \pounds33{,}302{,}909.40 \\
Best bound & \pounds32{,}592{,}032.86 \\
MIP gap & 2.13\% \\
Solve time & 600.05\,s \\
\bottomrule
\end{tabular}
\end{table}

\paragraph{Cost Breakdown.}

\begin{table}[h]
\centering
\caption{Part~2 cost breakdown}
\label{tab:part2-cost}
\begin{tabular}{lr}
\toprule
Cost Component & Amount (\pounds) \\
\midrule
Setup (one-time construction) & 5{,}524{,}000 \\
Operating (cumulative annual) & 4{,}379{,}400 \\
Transport: Warehouse $\to$ Customer & 19{,}133{,}394 \\
Transport: Supplier $\to$ Warehouse & 4{,}266{,}116 \\
\midrule
\textbf{Total} & \textbf{33{,}302{,}909} \\
\bottomrule
\end{tabular}
\end{table}

Transport costs dominate the total, accounting for approximately 70\% of the objective. The warehouse-to-customer leg alone represents 57\% of total cost, reflecting the high per-tonne-mile cost of 3.5t vans used for last-mile delivery.

\paragraph{Warehouse Build Schedule.}
The optimal solution opens 6 warehouses over the 10-period horizon:

\begin{table}[h]
\centering
\caption{Part~2 warehouse build schedule}
\label{tab:part2-warehouses}
\begin{tabular}{clcrr}
\toprule
ID & Postal District & Build Period & Capacity (kg) & Setup Cost (\pounds) \\
\midrule
315 & ML12 & 1 & 7{,}120{,}000 & 930{,}000 \\
411 & PH32 & 1 & 7{,}960{,}000 & 650{,}000 \\
42  & DD9  & 2 & 6{,}920{,}000 & 1{,}056{,}000 \\
183 & G84  & 3 & 6{,}600{,}000 & 1{,}350{,}000 \\
48  & DG4  & 6 & 7{,}520{,}000 & 734{,}000 \\
132 & FK20 & 6 & 7{,}320{,}000 & 804{,}000 \\
\bottomrule
\end{tabular}
\end{table}

\paragraph{Cumulative Open Warehouses by Period.}

\begin{table}[h]
\centering
\caption{Cumulative warehouse openings over the planning horizon (Part~2)}
\label{tab:part2-cumulative}
\begin{tabular}{cl}
\toprule
Period & Open Warehouses \\
\midrule
1 & ML12, PH32 \\
2 & ML12, PH32, DD9 \\
3 & ML12, PH32, DD9, G84 \\
4--5 & ML12, PH32, DD9, G84 \\
6--10 & ML12, PH32, DD9, G84, DG4, FK20 \\
\bottomrule
\end{tabular}
\end{table}

The solution exhibits a phased build-up: two warehouses are established immediately in Period~1, with additional warehouses added in Periods~2, 3, and~6 as demand grows. This gradual expansion reflects the trade-off between early setup investment and the cost of serving distant customers from fewer facilities.


% =============================================================================
\subsection{Part 3: Stochastic MECWLP with Uncertain Demands}
% =============================================================================

% -----------------------------------------------------------------------------
\subsubsection{Question (a): Deterministic Equivalent Formulation}
% -----------------------------------------------------------------------------

We extend the multi-period MECWLP to account for uncertain customer demands via scenario-based two-stage stochastic programming.

\paragraph{Uncertainty Model.}
The actual demand $\hat{b}_{i,p,t}$ in period $t$ lies within
\[
    \hat{b}_{i,p,t} \in \bigl[(1 - 0.1\,t)\, b_{i,p,t},\;(1 + 0.1\,t)\, b_{i,p,t}\bigr],
\]
where $b_{i,p,t}$ is the nominal demand. The deviation grows linearly: $\pm 10\%$ in year~1 up to $\pm 100\%$ in year~10. A set of $S$ equally likely scenarios $\mathcal{S} = \{1, \dots, S\}$ is given, with each scenario $s$ providing a demand realisation $b^s_{i,p,t}$.

\paragraph{Two-Stage Structure.}
\begin{itemize}
    \item \textbf{First stage} (here-and-now, scenario-independent): warehouse build decisions $y_{j,t}$ and cumulative open indicators $z_{j,t}$, which must be fixed before demand uncertainty is revealed.
    \item \textbf{Second stage} (wait-and-see, per scenario): allocation $x^s_{j,i,p,t}$ and supply flow $w^s_{k,j,t}$ decisions, which adapt to each demand scenario.
\end{itemize}

\paragraph{Additional Notation.}
\begin{itemize}
    \item $\mathcal{S} = \{1, \dots, S\}$: set of demand scenarios
    \item $\pi_s = 1/S$: probability of scenario $s$ (equally likely)
    \item $b^s_{i,p,t}$: demand of customer $i$ for product $p$ in period $t$ under scenario $s$
    \item $x^s_{j,i,p,t} \in [0,1]$: allocation fraction under scenario $s$
    \item $w^s_{k,j,t} \geq 0$: supply flow under scenario $s$
\end{itemize}

\paragraph{Deterministic Equivalent Formulation.}
The conservative (robust) formulation requires feasibility under \emph{all} scenarios simultaneously:
\begin{align}
\min \quad & \sum_{j \in \mathcal{J}} \sum_{t \in \mathcal{T}} f_j \, y_{j,t}
          + \sum_{j \in \mathcal{J}} \sum_{t \in \mathcal{T}} g_j \, z_{j,t} \notag \\
          & + \sum_{s \in \mathcal{S}} \pi_s \Biggl[
              \sum_{j \in \mathcal{J}} \sum_{i \in \mathcal{I}} \sum_{p \in \mathcal{P}} \sum_{t \in \mathcal{T}} c^{\text{down}}_{j,i} \, b^s_{i,p,t} \, x^s_{j,i,p,t}
            + \sum_{k \in \mathcal{K}} \sum_{j \in \mathcal{J}} \sum_{t \in \mathcal{T}} c^{\text{up}}_{k,j} \, w^s_{k,j,t}
          \Biggr]
\label{eq:part3-obj}
\end{align}
subject to, for all $s \in \mathcal{S}$:
\begin{align}
& z_{j,t} = \sum_{\tau=1}^{t} y_{j,\tau}
  & \forall\, j \in \mathcal{J},\; t \in \mathcal{T}
  \tag{C1} \\[4pt]
& \sum_{t \in \mathcal{T}} y_{j,t} \leq 1
  & \forall\, j \in \mathcal{J}
  \tag{C2} \\[4pt]
& \sum_{j \in \mathcal{J}} x^s_{j,i,p,t} = 1
  & \forall\, i \in \mathcal{I},\; p \in \mathcal{P},\; t \in \mathcal{T},\; s \in \mathcal{S}
  \tag{C3$'$} \\[4pt]
& \sum_{i \in \mathcal{I}} \sum_{p \in \mathcal{P}} b^s_{i,p,t} \, x^s_{j,i,p,t} \leq u_j \, z_{j,t}
  & \forall\, j \in \mathcal{J},\; t \in \mathcal{T},\; s \in \mathcal{S}
  \tag{C4$'$} \\[4pt]
& \sum_{k \in \mathcal{K}_p} w^s_{k,j,t} = \sum_{i \in \mathcal{I}} b^s_{i,p,t} \, x^s_{j,i,p,t}
  & \forall\, j \in \mathcal{J},\; p \in \mathcal{P},\; t \in \mathcal{T},\; s \in \mathcal{S}
  \tag{C5$'$} \\[4pt]
& \sum_{j \in \mathcal{J}} w^s_{k,j,t} \leq s_k
  & \forall\, k \in \mathcal{K},\; t \in \mathcal{T},\; s \in \mathcal{S}
  \tag{C6$'$} \\[4pt]
& y_{j,t} \in \{0,1\},\; z_{j,t} \in \{0,1\}
  & \forall\, j, t
  \notag \\
& x^s_{j,i,p,t} \in [0,1],\; w^s_{k,j,t} \geq 0
  & \forall\, j, i, p, t, k, s
  \notag
\end{align}

\noindent\textbf{Key differences from Part~2:}
\begin{itemize}
    \item The objective function computes the \emph{expected} transport cost over all scenarios, weighted by $\pi_s = 1/S$. Setup and operating costs are scenario-independent.
    \item Constraints (C1) and (C2) remain scenario-independent: the warehouse infrastructure must be decided before uncertainty is revealed.
    \item Constraints (C3$'$)--(C6$'$) are duplicated for \emph{each} scenario $s$: the allocation and supply decisions adapt to each demand realisation, but the shared warehouse structure must support feasibility under all scenarios simultaneously.
    \item This is a \emph{conservative} formulation: it guarantees that all customer demand can be met regardless of which scenario materialises.
\end{itemize}


% -----------------------------------------------------------------------------
\subsubsection{Question (b): Implementation and Results}
% -----------------------------------------------------------------------------

The model was implemented with $S = 20$ scenarios (out of the 100 available). The candidate set was filtered to 75 (top-5 per postal area) to make the problem tractable.

\paragraph{Model Size.}

\begin{table}[h]
\centering
\caption{Model dimensions for Part~3}
\label{tab:part3-dimensions}
\begin{tabular}{lr}
\toprule
Component & Count \\
\midrule
First-stage binary variables ($y_{j,t}$) & 750 \\
First-stage open indicators ($z_{j,t}$) & 750 \\
Second-stage allocation variables ($x^s_{j,i,p,t}$) & 900{,}000 \\
Second-stage supply flows ($w^s_{k,j,t}$) & 795{,}000 \\
\textbf{Total variables} & \textbf{1{,}696{,}500} \\
Total constraints & 98{,}425 \\
Number of scenarios & 20 \\
\bottomrule
\end{tabular}
\end{table}

\paragraph{Solution Quality.}

\begin{table}[h]
\centering
\caption{Part~3 solution summary}
\label{tab:part3-solution}
\begin{tabular}{lr}
\toprule
Metric & Value \\
\midrule
Status & Feasible \\
Objective value & \pounds37{,}980{,}845 \\
Best bound & \pounds34{,}345{,}080 \\
MIP gap & 9.57\% \\
Solve time & 601.71\,s \\
\bottomrule
\end{tabular}
\end{table}

\paragraph{Cost Breakdown.}

\begin{table}[h]
\centering
\caption{Part~3 cost breakdown}
\label{tab:part3-cost}
\begin{tabular}{lr}
\toprule
Cost Component & Amount (\pounds) \\
\midrule
Setup (one-time construction) & 9{,}018{,}000 \\
Operating (cumulative annual) & 5{,}507{,}600 \\
$\mathbb{E}$[Transport: Warehouse $\to$ Customer] & 18{,}752{,}387 \\
$\mathbb{E}$[Transport: Supplier $\to$ Warehouse] & 4{,}702{,}858 \\
\midrule
\textbf{Total} & \textbf{37{,}980{,}845} \\
\bottomrule
\end{tabular}
\end{table}

Compared to the deterministic solution, setup costs increase by 63\% (\pounds9.02M vs.\ \pounds5.52M) and operating costs increase by 26\% (\pounds5.51M vs.\ \pounds4.38M), reflecting the need for more warehouse capacity to handle worst-case demand scenarios. The expected downstream transport cost is slightly lower (\pounds18.75M vs.\ \pounds19.13M), as the additional warehouses provide better geographic coverage, while the upstream transport cost increases (\pounds4.70M vs.\ \pounds4.27M) due to serving a larger and more distributed warehouse network.

\paragraph{Warehouse Build Schedule.}
The stochastic solution opens 12 warehouses (vs.\ 6 in the deterministic case):

\begin{table}[h]
\centering
\caption{Part~3 warehouse build schedule}
\label{tab:part3-warehouses}
\begin{tabular}{clcrr}
\toprule
ID & Postal District & Build Period & Capacity (kg) & Setup Cost (\pounds) \\
\midrule
19  & AB36 & 1 & 7{,}880{,}000 & 650{,}000 \\
315 & ML12 & 1 & 7{,}120{,}000 & 930{,}000 \\
400 & PH18 & 1 & 7{,}480{,}000 & 748{,}000 \\
48  & DG4  & 3 & 7{,}520{,}000 & 734{,}000 \\
183 & G84  & 4 & 6{,}600{,}000 & 1{,}350{,}000 \\
235 & IV51 & 5 & 7{,}760{,}000 & 678{,}000 \\
351 & PA36 & 6 & 7{,}920{,}000 & 650{,}000 \\
54  & DG10 & 7 & 7{,}960{,}000 & 650{,}000 \\
411 & PH32 & 7 & 7{,}960{,}000 & 650{,}000 \\
58  & DG14 & 8 & 7{,}960{,}000 & 650{,}000 \\
47  & DG3  & 9 & 7{,}720{,}000 & 678{,}000 \\
345 & PA30 & 10 & 8{,}000{,}000 & 650{,}000 \\
\bottomrule
\end{tabular}
\end{table}

\paragraph{Cumulative Open Warehouses by Period.}

\begin{table}[h]
\centering
\caption{Cumulative warehouse openings over the planning horizon (Part~3)}
\label{tab:part3-cumulative}
\begin{tabular}{ccl}
\toprule
Period & Count & Warehouses \\
\midrule
1--2 & 3 & AB36, ML12, PH18 \\
3 & 4 & $+$ DG4 \\
4 & 5 & $+$ G84 \\
5 & 6 & $+$ IV51 \\
6 & 7 & $+$ PA36 \\
7 & 9 & $+$ DG10, PH32 \\
8 & 10 & $+$ DG14 \\
9 & 11 & $+$ DG3 \\
10 & 12 & $+$ PA30 \\
\bottomrule
\end{tabular}
\end{table}

The stochastic model requires continuous warehouse expansion throughout the entire horizon, with new facilities added in 8 out of 10 periods. This contrasts sharply with the deterministic model, which stabilises after Period~6. The growing uncertainty band ($\pm 10t\%$) forces the solution to build additional capacity in later periods to accommodate potential demand surges under the most extreme scenarios.

\paragraph{Comparison: Deterministic vs.\ Stochastic.}

\begin{table}[h]
\centering
\caption{Comparison of deterministic and stochastic solutions}
\label{tab:comparison}
\begin{tabular}{lrr}
\toprule
& Part~2 (Deterministic) & Part~3 (Stochastic) \\
\midrule
Objective (\pounds) & 33{,}302{,}909 & 37{,}980{,}845 \\
MIP gap & 2.13\% & 9.57\% \\
Total variables & 506{,}000 & 1{,}696{,}500 \\
Constraints & 27{,}970 & 98{,}425 \\
Binary variables & 8{,}800 & 1{,}500 \\
Warehouses opened & 6 & 12 \\
Setup cost (\pounds) & 5{,}524{,}000 & 9{,}018{,}000 \\
Operating cost (\pounds) & 4{,}379{,}400 & 5{,}507{,}600 \\
\bottomrule
\end{tabular}
\end{table}

The stochastic formulation incurs approximately \pounds4.68M (14\%) additional cost to guarantee demand satisfaction under all 20~scenarios. The cost of robustness is concentrated in the infrastructure components: the number of warehouses doubles, and setup plus operating costs increase from \pounds9.90M to \pounds14.53M. The larger MIP gap (9.57\% vs.\ 2.13\%) reflects the increased computational difficulty of the scenario-based formulation, which multiplies the second-stage variables and constraints by the number of scenarios.


% -----------------------------------------------------------------------------
\subsubsection{Question (c): Less Conservative Formulation}
% -----------------------------------------------------------------------------

The formulation in Question~(a) is maximally conservative: it requires demand satisfaction in \emph{every} scenario. A less conservative alternative is the \textbf{penalty-based recourse formulation}, which allows unmet demand at a cost.

\paragraph{Idea.} Introduce non-negative \emph{shortage variables} $\delta^s_{i,p,t} \geq 0$ representing unmet demand for customer $i$, product $p$, in period $t$ under scenario $s$. The demand satisfaction constraint \eqref{eq:c3} is relaxed to:
\[
    \sum_{j \in \mathcal{J}} x^s_{j,i,p,t} + \frac{\delta^s_{i,p,t}}{b^s_{i,p,t}} = 1
    \qquad \forall\, i, p, t, s
\]
and a penalty cost $q \cdot \delta^s_{i,p,t}$ is added to the objective for each unit of unmet demand, where $q > 0$ is the unit penalty cost (e.g.\ lost-sales cost, goodwill penalty, or contractual fine). The objective becomes:
\begin{align*}
\min \quad & \text{(setup + operating)} + \sum_{s \in \mathcal{S}} \pi_s \Biggl[
    \text{(transport cost under $s$)} + q \sum_{i \in \mathcal{I}} \sum_{p \in \mathcal{P}} \sum_{t \in \mathcal{T}} \delta^s_{i,p,t}
\Biggr]
\end{align*}

\paragraph{Advantages.}
\begin{itemize}
    \item Reduces the total infrastructure cost by not requiring capacity for the most extreme demand realisations.
    \item The penalty parameter $q$ provides a direct lever to control the trade-off between cost and service level: higher $q$ approaches the fully conservative solution, while lower $q$ allows more flexibility.
    \item Computationally easier: relaxing hard constraints can improve solver convergence and reduce the MIP gap.
\end{itemize}

\paragraph{Alternative approaches.} Other less conservative formulations include:
\begin{itemize}
    \item \textbf{Chance constraints}: require $\Pr\bigl[\text{demand satisfied}\bigr] \geq 1 - \epsilon$ for each customer, allowing a small fraction of scenarios to be infeasible.
    \item \textbf{CVaR-based objective}: minimise cost subject to a bound on the conditional value-at-risk of unmet demand, providing tail-risk control.
\end{itemize}
